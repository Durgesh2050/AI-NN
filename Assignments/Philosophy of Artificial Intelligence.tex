\documentclass{article}
\usepackage[utf8]{inputenc}

\title{Philosophy of Artificial Intelligence}
\author{Submitted by Durgesh Kumar}
\date{18th July 2021}
\begin{document}

\maketitle

\section{Introduction}
Artificial intelligence is a branch of the philosophy of technology that explores artificial intelligence and its implications for knowledge and understanding of intelligence, ethics, consciousness, and free will. some scholars argue that the AI community’s dismissal of philosophy is detrimental. The philosophy of artificial intelligence attempts to answer such questions as follows: Can a machine act intelligently? CAN IT SOLVE ANY PROBLEM that a person would solve by thinking? Are human intelligence and machine intelligence the same?
Is the human brain essentially a computer? Can machines have a mind, mental states, and consciousness in the same sense that a human being can? Can they feel how things are?
\section{Can a machine display general intelligence? }
Is it possible to create a machine that can solve all the problems humans solve using their intelligence? This question defines the scope of what machines could do in the future and guides the directions of Ai research. To answer this question, it does not matter whether a machine is really thinking (a person thinks) or is just acting like it is thinking.
Alan Turing reduced the problem of the defining intelligence to a simple question about conversation. He suggests that if a machine can answer any question using the same words that an ordinary person would, then we may call that machine intelligent. Modern version of his experimental design would use an online chart room, where one of the participants is a real person and one of them is a computer program. The program passes the test if no one can tell which of the two participants is human.
The argument that a machine can display general intelligence was first introduced as early as 1943 and vividly described by Hans Moravec in 1988. Some critics of AI such as Hubert Dreyfus and John Searle argue that anything can be stimulated by a computer; thus, any process at all can technically be considered “computation”.
Modern AI, based on statistics and mathematical optimization, does not use the high level “symbol processing” that they discussed. Gödel’s theorems do not lead to any argument that humans have mathematical reasoning capabilities beyond what a machine could ever duplicate. The arguments had been anticipated by Alan Turing in his 1950 paper computing machinery and intelligence. New models are being developed to capture our unconscious skills at perceptions and attention. These include neural nets, evolutionary algorithms and so on.
\section{Can a machine have a mind, consciousness, and mental states?}
John Searle argued that even if we had a computer program that acted exactly like a human mind, there would still be a difficult philosophical question that needed to be answered. John Searle’s two positions do not directly answer the question “can a machine display general intelligence?”
Science fiction writers use it to describe some essential property that makes us human: intelligence, desires, will, insight, pride and so on. John Searle asks us to consider a thought experiment: suppose we have written a computer program that passes the Turing test and demonstrates general intelligent action. He concludes that the Chinese room, or any other physical symbol system, cannot have a mind. Searle’s argument is just a version of the problem of other minds, applied to machines. The question is whether “consciousness” exist.
\section{Is thinking a kind of computation?}
Computationalism claims that the relationship between mind and brain is similar to that between a running program and a computer. If the human brain is a kind of computer then computers can be both intelligent and conscious, answering both the practical and philosophical questions of AI. The idea has philosophical roots in Hobbes and Leibniz, Hume and Kant. 
Other related questions like Can a machine have emotions? Can a machine be self-aware? Can a machine be original or creative? Can a machine be benevolent or hostile? Can a machine imitate all human characteristics? 
\section{views on the role of philosophy}
Some philosophers argue that the role of philosophy in AI is underappreciated. Physicist David Deutsch argues that without an understanding of philosophy or its concepts, AI development would suffer from a lack of progress. The Sandford Encyclopaedia of Philosophy argues that the AI community’s dismissal of philosophy is determined. 
\end{document}
